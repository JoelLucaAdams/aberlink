\documentclass[11pt,fleqn,twoside]{article}
\usepackage{makeidx}
\makeindex
\usepackage{palatino} %or {times} etc
\usepackage{plain} %bibliography style
\usepackage{amsmath} %math fonts - just in case
\usepackage{amsfonts} %math fonts
\usepackage{amssymb} %math fonts
\usepackage{lastpage} %for footer page numbers
\usepackage{fancyhdr} %header and footer package
\usepackage{mmpv2}
%\usepackage{url}
\usepackage{hyperref}

\usepackage{cite}

\begin{document}

\name{Joel Adams}
\userid{joa38}
\projecttitle{Discord bot for verification and attendance linked to OpenID Connect}
\projecttitlememoir{Discord link to OpenID Connect} %same as the project title or abridged version for page header
\reporttitle{Project Outline}
\version{1.0}
\docstatus{Release} % change to Release when submitting final version
\modulecode{CS39440}
\degreeschemecode{GG4R}
\degreeschemename{Computer Science and Artificial Intelligence}
\supervisor{Neal Snooke} 
\supervisorid{nns}

%optional - comment out next line to use current date for the document
\documentdate{26th January 2021}
\mmp

\setcounter{tocdepth}{3} %set required number of level in table of contents


\section{Project description}
The Discord\cite{discord} link to OpenID Connect\cite{OpenID} project will develop a website where users can login using their university credentials and create a link to their Discord account. From the website users can monitor which university servers they are in. This project will also contain a front-end Discord bot to verify users and monitor attendance. \bigskip

A website will be created called discord.dcs.aber.ac.uk and will be built using the Python Django\cite{Django} framework and can be accessed using OpenID Connect login system. After the user signs in for the first time they will be requested to sign in with Discord using OAuth2 to link their accounts together (this only happens once). Their uni email and  Discord account will be saved in a relational database using PostgreSQL\cite{psql} (PSQL) and stored there for use in the bot. \bigskip

The website will offer a dashboard that displays information on the servers that the user is currently in (module servers only) by using the bot to cross check if the user's Discord ID exists in a specific server. If the user is not present inside of a module server the bot will create an invite link and make a clickable hyperlink on the website for the student to use to join the server. If the logged-in user is a lecturer then they will be able to see the servers for which they have an administrative role and view all the users that are currently in those servers. There will also be an interface for lecturers to provision module servers where they can remove any of last year's students, all the channels and re-create fresh channels or create a new server entirely using a server template. \bigskip

This project builds on the Discord bots 'AberVerify'\cite{AberVerify} and 'I am here'\cite{I_am_here} by integrating them into a singular bot in Python using the Discord.py\cite{discord.py} library. The 'AberVerify' bot is responsible for verifying the identity of students in Discord by cross checking their IDs with their aber.ac.uk email addresses. The new system will use a PSQL database to check if a Discord user's ID exists and give them the verified role once they join the server. The 'I am here' bot is responsible for acknowledging the attendance of a student in a practical by having the student type a command in Discord and the bot will send an email to the lecturer containing the student's email. The new system will wait until the student sends the command to confirm their presence in the practical and then send the request to the university servers to mark the student as present during the practical. It will also send them a DM on Discord with a timestamp and module code as a confirmation receipt of the update. \bigskip

The end goal of this project is to provide a piece of software that can be run and maintained by the DCS easily and used for Discord servers in the foreseeable future. Therefore the project will have substantially more documentation to explain how each component of the software works.

\section{Proposed tasks}
\begin{itemize}
	\item \textbf{Research and discussion with IS/CS-support.} First step is a discussion with IS and CS-support about access that can be granted to the different APIs including the attendance API, the OpenID connect configuration and the Discord OAuth2 login. A container in the Department of Computer Science (DCS) will also be configured to allow for a live server to test on. A database using PSQL will be established in DCS as well. This stage is already mostly completed as it was needed for approval of this document.
	 
	\item \textbf{Version control, documentation and setup.} A container called \url{http://mmp-joa38.dcs.aber.ac.uk/} has been set up so builds will be run on it instead of locally. The end goal of this project is to create a finalised product that can be used in real DCS Discord servers so documentation and version control will be done on \url{https://gitlab.dcs.aber.ac.uk}. A journal of the events during this project will also be located in the repository using markdown files for each week.
	 
	\item \textbf{Creation of Python back-end for website and the database.} The website will be built using the Python framework Django as it scales well for the size of the project and is easily configurable. Access to the PSQL database must be configured to allow for data writing and reading in Django.
	
	\item \textbf{Re-writing `AberVerify` and `I am here` into single Python bot.} The bot will perform the previously stated tasks of verification using the login system with PSQL and it will update attendance by sending and receiving responses from the university.
	
	\item \textbf{Interface for lecturers and students on discord.dcs.aber.ac.uk.} Allows students to view which servers they are in and links to join module servers they are not in. Lecturers can view servers for which they have administrative privileges and can configure student's roles online by creating a new table that links students to roles in the server.
	
	\item \textbf{Resource links on the Discord bot.} If there is remaining time in the project more features will be added to the bot including links to resources in the department and across the university (e.g. link to blackboard, student record, abersu, etc.). The bot may also contain a command to display users' Discord usernames and Aber usernames to confirm that the link has worked correctly (this would only display the current users' information and not other users' identities).
\end{itemize}


\section{Project deliverables}
\begin{itemize}
	\item \textbf{Website hosted on discord.dcs.aber.ac.uk.} As seen in the previous section this website will have a login portal using OpenID Connect, link accounts to Discord and have a user interface for both students and lecturers.
	
	\item \textbf{Discord bot for front-end.} A bot that will be responsible for marking students' attendance and verifying students. It will talk to a back-end PSQL database.
	
	\item \textbf{Wiki, unit tests and documentation for software.} For future maintainability of the product substantial wiki and documentation must be provided to allow developers to easily understand and modify the code and it's functionality. Functional unit tests will be provided to allow for checking if the code is performing as intended.
	
	\item \textbf{Final report.} The final report will contain a thorough brief of the project and the process by which to update and maintain it.
\end{itemize}


\nocite{*} % include everything from the bibliography, irrespective of whether it has been referenced.

% the following line is included so that the bibliography is also shown in the table of contents. There is the possibility that this is added to the previous page for the bibliography. To address this, a newline is added so that it appears on the first page for the bibliography.
\newpage
\addcontentsline{toc}{section}{Initial Annotated Bibliography}

\bibliographystyle{IEEEannotU}
\renewcommand{\refname}{References}
\bibliography{mmp} % References file

\end{document}
