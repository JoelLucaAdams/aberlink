\chapter{Third-Party Code and Libraries}

% If you have made use of any third party code or software libraries, i.e. any code that you have not designed and written yourself, then you must include this appendix. 

% As has been said in lectures, it is acceptable and likely that you will make use of third-party code and software libraries. If third party code or libraries are used, your work will build on that to produce notable new work. The key requirement is that we understand what your original work is and what work is based on that of other people. 

% Therefore, you need to clearly state what you have used and where the original material can be found. Also, if you have made any changes to the original versions, you must explain what you have changed. 

% The following is an example of what you might say. 

% Apache POI library - The project has been used to read and write Microsoft Excel files (XLS) as part of the interaction with the client's existing system for processing data. Version 3.10-FINAL was used. The library is open source and it is available from the Apache Software Foundation 
% \cite{apache_poi}. The library is released using the Apache License 
% \cite{apache_license}. This library was used without modification. 

% Include as many declarations as appropriate for your work. The specific wording is less important than the fact that you are declaring the relevant work.
I declare that all the following libraries are unmondified.

\begin{itemize}
    \item \textbf{Pipenv} \cite{pipenv} - This is a Python package manager that is used throughout this project to create and maintain virtual environments. It uses Pipfiles (text files) to store a list of dependencies that are then downloaded and stored in a virtual environment. In this project I have two Pipfiles; one for the website (\verb|src\AberLinkAuthentication|) and one for the Discord bot (\verb|src\AberLinkDiscord|). This library uses the MIT license so is free to use.
    \item \textbf{Django} \cite{Django} - This is a Python web framework used for building websites. It is used to display the webpages, sign users in and help users and staff manage their data. This library is open source and provided by the \href{https://www.djangoproject.com/trademarks/#:~:text=Django%20is%20an%20Open%20Source,the%20use%20of%20a%20trademark.}{Django Software Foundation}.
    \item \textbf{Apache2} \cite{apache2} - This Debian \cite{debian} package is responsible for hosting a HTTP web server. In this case it hosts the Django \cite{Django} framework so it can be accessed on the web. This library is released under the \href{https://www.apache.org/licenses/LICENSE-2.0}{Apache License, Version 2.0}.
    \item \textbf{libapache2-mod-auth-openidc} \cite{libapache2-mod-auth-openidc} - Debian \cite{debian} package used to authenticate users when they first connect to the site. It authenticates users against their Aberystwyth University account and then creates a session where they can access content on the website. This library is released under the \href{https://static.fsf.org/nosvn/directory/fdl-1.3-standalone.html}{GNU Free Documentation License}
    \item \textbf{certbot} \cite{certbot} - This package was only briefly required to setup HTTPS on the website using Let's Encrypt \cite{letsencrypt}. Once SSL certificates have been made cerbot then runs a cronjob in the background to update it. This library is licensed under the \href{https://certbot.eff.org/faq#what-are-the-licenses-for-certbot-and-this-website}{Apache 2.0 license}
    \item \textbf{psycopg2} \cite{psycopg2} - This library is used for talking to PostgreSQL databases in Python. The version used here is called psycopg2-binary and it requires two additional dependencies to be built libpq-dev \cite{libpq-dev} and python-dev \cite{python-dev}. This library is licensed under the \href{https://www.psycopg.org/license/}{GNU Lesser General Public License}.
    \item \textbf{Discord.py} \cite{discord.py} - This Python library is used to create and write the Discord bot. It is basically an interface that allows me to write bots without making API requests straight to Discord. This library is licensed under the \href{https://github.com/Rapptz/discord.py/blob/master/LICENSE}{MIT License}.
    \item \textbf{Requests} \cite{requests} - It is used to make HTTP requests to the web. In this project it is used in the website to get information from the Discord login and in the Discord bot it is used to send and receive data from the attendance API endpoint \href{https://integration.aber.ac.uk/joa38/submit.php}{https://integration.aber.ac.uk/joa38/submit.php}. This library is licensed under the \href{https://pypi.org/project/requests/}{Apache Software License (Apache 2.0)}
    \item \textbf{Pytz, SQLParse \& ASGIref} \cite{pytz} \cite{sqlparse} \cite{asgiref} - These are dependencies required to work with Django. pytz helps with managing timezones and timezone data. SQLParse handles SQL queries, formatting and splitting. Asgiref is an interface used to talk between Apache2 \cite{apache2} and Django \cite{Django} for web server hosting and frameworks. These libraries are all licensed under the MIT License.
\end{itemize}